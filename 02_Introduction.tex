%%%% This file includes the introduction %%%%

\section{Introduction}

Agriculture must strongly increase its production to feed the nine-billion people predicted by mid-century, while minimizing its enviromental impact. In this context, precise and efficient information about the state of different crops can be used to develop commercial plans, regulate internal stocks and to perform customized management decisions based on varying soil types, landscape position and land usage history. In the last years, remote sensing imagery has increasingly been used for this task because it is a cost-efficient means for gathering timely, detailed and reliable information over large areas with short revisit periods \cite{thenkabail2015land,leite2011hidden}.

Crop recognition is challenging because some fields are covered with different types of crops during the year and such sequence could be governed by phenological, ecologic or economic reasons. Thus, agricultural areas are characterized by their temporal dynamics and also by their typical spatial patterns \cite{lohmann2008multi,waske2009classifier}. 

A commonly used method consists of stacking the multi-temporal sequence of images together and train a classifier such as RF or SVM using information from the individual pixels. This approach has the disadvantage that it ignores spatial context. Object-based approaches perform segmentation on the input data and classify the resulting segments. Although segmentation considers the image's spatial context, it only takes into account the input data and ignores the semantic information. Probabilistic graphical such as Conditional Random Fields are able to capture spatio-temporal context and have been used for crop recognition \cite{achanccaray2017spatial}, although it still requires features to be manually extracted. In recent years, deep learning models have made breakthroughs in several fields such as speech recognition and computer vision. In the field of remote sensing, deep learning models have become a new way to solve old problems and have achieved state-of-the-art in multiple applications \cite{audebertdeep}. In particular, recurrent networks are designed to model temporal data while convolutional networks are useful for understanding spatial context. In \cite{jose}, recurrent and convolutional networks were successfully applied to crop recognition. Likewise, fully convolutional networks (FCN) have became state-of-the-art in semantic segmentation tasks while improving inference time \cite{long2015fully}. In \cite{} a FCN was applied for the multi-temporal crop recognition task. In \cite{castro2017comparative} CNNs were also applied to this problem. 

As a  combination of previous ideas, convolutional LSTM networks (ConvLSTM) are able to model both spatial and temporal dependencies by replacing the LSTM input-to-state and state-to-state layers with convolutions \cite{xingjian2015convolutional}. This work compares different deep learning architectures based in recurrent and fully convolutional networks, and proposes a ConvLSTM-based network for the problem of multi-temporal crop recognition. Models were evaluated using two study areas with very different weather conditions: First one is located in Hanover, Germany with a template climate. The second area corresponds to a sub-tropical region in Campo Verde municipality, Brazil.

The remainder of this paper is organized as follows: Section II briefly explains the theoretical fundamentals to understand the models. Section III shows the used model architectures. In section IV, the study areas and the experimental protocol are discussed. Then results are discussed in Section V. Finally, conclusions are presented in Section VI.