%%%% This file includes the introduction %%%%

\section{Introduction}

Agriculture must strongly increase its production to feed the nine-billion people predicted by mid-century, while minimizing its environmental impact. In this context the demand for efficient, comprehensive and precise agriculture intelligence has seen an important increase during past years. In particular, crop production information can be used to develop commercial plans, regulate internal stocks and to perform customized management decisions based on varying soil types, landscape position and land usage history. Remote sensing imagery has increasingly been used for this task because it is a cost-efficient means for gathering timely, detailed and reliable information over large areas with short revisit periods \cite{thenkabail2015land,leite2011hidden}.

Crop recognition is challenging because some fields are covered with different types of crops during the year and such practice could be influenced by multiple reasons including phenological, ecologic or economic changes. Thus, agricultural areas are characterized by their temporal dynamics as well as their typical spatial patterns \cite{lohmann2008multi,waske2009classifier}. 

A commonly used method consists of stacking the multi-temporal sequence of images together and train a classifier such as Random Forest (RF) using information from each individual pixel, neglecting any spatial relationship between neighbor pixels. Object-based approaches perform image segmentation before the classification step, where segments instead of pixels are taken to train the classifier. Although segmentation considers the image's spatial context, it lacks of semantic information. Probabilistic graphical models such as Conditional Random Fields are able to capture spatio-temporal context and have been used for crop recognition \cite{achanccaray2017spatial}, yet it is necessary to study what kind of hand-crafted features will be employed. 
In recent years, deep learning models have made breakthroughs in several fields such as speech recognition and computer vision. In the field of remote sensing, these models have become a new way to solve old problems and have achieved state-of-the-art in multiple applications \cite{audebertdeep}. In particular, Recurrent Neural Networks (RNN) are designed to model temporal data while Convolutional Networks (CNN) are useful for understanding spatial context. In \cite{rnnjose}, recurrent and convolutional networks were successfully applied to crop recognition. Likewise, Fully Convolutional Networks (FCN) have became state-of-the-art in semantic segmentation tasks while improving inference time \cite{long2015fully}. In \cite{fcnlaura} a FCN was applied for the multi-temporal crop recognition task. As a combination of previous ideas, convolutional LSTM networks (ConvLSTM) are designed to model both spatial and temporal context by replacing the LSTM input-to-state and state-to-state layers with convolutions \cite{xingjian2015convolutional}. 

This work compares some of the most promising deep learning architectures for multi-temporal crop recognition: Fully convolutional and recurrent networks. Besides, it proposes a new architecture based on the ConvLSTM cell. To our knowledge, this is the first time the ConvLSTM is used for this particular problem. Models were evaluated over two study areas with very different weather conditions: First is located in Hanover, Germany with a template climate. Second area corresponds to a sub-tropical region in Campo Verde municipality, Brazil. 

The remainder of this paper is organized as follows: Section II briefly explains the theoretical fundamentals to understand the models. Section III shows the used model architectures. In section IV, the study areas and the experimental protocol are discussed. Then results are discussed in Section V. Finally, conclusions are presented in Section VI.