%%%% This file includes the conclusions %%%%

\section{Conclusions}

In this work, some of the most successful deep learning architectures for multi-temporal crop recognition were implemented; tailored to the specific dataset requirements; and compared. Furthermore, a new architecture based on ConvLSTM was proposed. Their performance were assessed over two very different datasets. %located in Hanover, Germany and Campo Verde, Brazil. 

One of the main contributions of this work is the classification improvement for IS, LSTM-PC and ConvLSTM-PC achieved by finding their most appropiate input sizes. Results indicate that the abilities to represent spatial semantics from FCN-PC might be better harnessed in areas with larger crop tile sizes like the ones from Campo Verde. Likewise, the spatio-temporal modeling properties from ConvLSTM might be more relevant for datasets with larger time sequences. Future works will focus in combining both architectures into a fully convolutional recurrent network.
