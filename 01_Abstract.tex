%%%% This file includes the abstract %%%%

 
\section*{\textit {Abstract}} %%%% (DO NOT CHANGE) %%%%


\hspace{-1.5mm}\textit{ %%%% (DO NOT CHANGE) %%%%
%%%% Add the abstract below %%%%
With population and food consumption continuously growing, the past years have seen an increased demand for efficient agricultural crop monitoring systems. Crop dynamics are inherently complex and require both spatial and temporal context to be modeled for adequate mappings. The increasing availability of timely, precise and cost-effective Remote Sensing data makes it ideal for this type of multi-temporal analysis. On the other hand, deep learning techniques have made great breakthroughs in the field of computer vision. This work performs a comparative analysis over some of the most promising deep learning architectures for the task of multi-temporal crop recognition: Fully convolutional and recurrent networks. Experiments were carried out in a temperate region near Hanover, Germany and in a sub-tropical region with more complex temporal dynamics in Campo Verde, Brazil. Results show that the implemented models achieve state-of-the-art results.
}
\\
% , with the fully convolutional and convolutional recurrent networks achieving the best performance for Campo Verde and Hanover.
%%%% Add the Key words below %%%%
\textit{\textbf{Key words --} %%%% (DO NOT CHANGE) %%%%
Convolutional LSTM, Fully Convolutional Neural Network, Recurrent Neural Networks, Multi-Temporal Crop Recognition. %%%% (ADD KEY WORDS) %%%%
}