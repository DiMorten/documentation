%%%% This file includes the results %%%%

\section{Results}



Results are shown in Table \ref{tab:results}. The baseline model \textit{IS} achieved significantly higher results than the ones from previous works in Campo Verde dataset. This can be explained because a GLCM window size selection was performed. Even so, the deep learning networks outperformed this baseline in terms of OA and AA for all datasets. 

In general, the FCN-PL architecture performed the best for Campo Verde dataset. Because of its larger tile sizes, this study area might have better benefited from the spatial modeling that the FCN-PL is able to perform. On the other hand, the amount of temporal images in Hanover dataset is more than three times larger when compared to Campo Verde. The fact that the ConvLSTM performed better in this context indicates that its sequence modeling capabilities might be better harnessed with larger time sequences.

Although both recurrent networks presented similar overall accuracy values, the ConvLSTM-based model outperformed its counterpart in terms of average accuracy for all datasets. The fact that minority classes were better represented by ConvLSTM-PC could indicate that this model is more capable of encoding the additional information from the image data augmentation process.   \vspace{0.5cm}

%Both LSTM-PC and ConvLSTM-PC outperformed the results obtained in \ref{} for Campo Verde. This might be the result of certain architecture modifications such as a larger input image size and an extra intermediate layer. Finally, all the deep models obtained a better performance than the basic image stacking approach. 

\begin{table}[h!]
\centering
\caption{Results for Campo Verde and Hanover datasets. Metrics are Overall Accuracy (OA) and Average Accuracy (AA)}
\label{tab:results}
\begin{tabular}{|l|c|c|c|c|c|c|}
\hline
\multicolumn{1}{|c|}{\multirow{2}{*}{\textbf{Dataset}}}         & \multicolumn{4}{c|}{\textbf{Campo Verde}}                                           & \multicolumn{2}{c|}{\multirow{2}{*}{\textbf{Hanover}}}              \\ \cline{2-5}
\multicolumn{1}{|c|}{}                                          & \multicolumn{2}{c|}{\textbf{Sequence 1}} & \multicolumn{2}{c|}{\textbf{Sequence 2}} & \multicolumn{2}{c|}{}                                               \\ \hline
\multicolumn{1}{|c|}{\textbf{Layer}}                            & \textbf{OA}        & \textbf{AA}         & \textbf{OA}          & \textbf{AA}       & \multicolumn{1}{l|}{\textbf{OA}} & \multicolumn{1}{l|}{\textbf{AA}} \\ \hline
\textbf{FCN-PL}                                                 & \textbf{81}        & 75.6                & \textbf{73.9}        & \textbf{69}       & 91.9                             & 88.5                             \\ \hline
\textbf{\begin{tabular}[c]{@{}l@{}}ConvLSTM-\\ PC\end{tabular}} & 80.5               & \textbf{75.9}       & 70.4                 & 66.4              & \textbf{93.7}                    & \textbf{90.2}                    \\ \hline
\textbf{LSTM-PC}                                                & 80.1               & 74                  & 72                   & 65                & 91.9                             & 85.9                             \\ \hline
\textbf{IS(GLCM)}                                               & 79.1               & 68.1                & 71.1                 & 65.9              & 86.1                             & 77.4                             \\ \hline
\end{tabular}
\end{table}