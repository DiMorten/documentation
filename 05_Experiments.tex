\section{Experiments}

%\begin{figure}[t!]
%\label{fig:campoverde}
%\centering
%\includegraphics[scale=0.22]{figs2/CampoVerde.pdf}
%\caption{Campo Verde study area.}
%\label{fig:labels}
%\end{figure}

%\begin{figure}[t!]
%\label{fig:hanover}
%\centering
%\includegraphics[scale=0.18]{figs2/Hanover.pdf}
%\caption{Hanover study area.}
%\label{fig:labels}
%\end{figure}
\subsection{Study Areas}
In an attempt to evaluate the robustness of the methods, two study areas with different agricultural practices and crop dynamics were selected: The first area is located in the surroundings of Hanover city, in Germany. It has an extension of 1728 $km^2$ and consists of a sequence of 24 SAR images, dual-polarized from Sentinel-1 satellite taken from October 2014 to October 2016. One particular characteristic of this area, located in a temperate region, is the absence of crop rotation, which implies that a parcel has the same crop class throughout the entire agricultural year \cite{bargiel2017new}.

Second area is located in Campo Verde municipality from the state of Mato Grosso, Brazil with an extension of 4782 $km^2$. It consists of a sequence of 14 SAR images from Sentinel-1, dual polarized, and taken from October 2015 to July 2016. In contrast with the first area, it is located in a tropical region with crop rotation and different agricultural practices, making its multi-temporal behaviour highly dynamic and challenging \cite{sanches2018campo}.

\subsection{Experimental Protocol}
 % Random Forest classifier that assigns the class label probabilities for a given pixel. Particularly the values of correlation, homogeneity, mean and variance from GLCM matrix in four directions (0, 45, 90 and 135 degrees) were used as hand-crafted features. 
In order to assess the influence of multi-temporal information for crop recognition, the last image from a given sequence was classified using the past information available from all the images from that sequence.

In each of the studied areas, 50\% of the labeled data was used for training and the remaining 50\% for testing. Standard normalization was applied to the input images. In the case of Campo Verde database, two main sequences were separately studied due to their significant differences in crop distribution (See Figure \ref{fig:class_distr}). Sequence 1 consists of images taken from October 2015 to February 2016, while Sequence 2 comprises images from March 2016 to July 2016.

The required amount of spatial information differed between studied areas due to their specific crop parcel sizes. Thus, an input patch size of $5 \times 5$ was used in Hanover and of $15 \times 15$ in Campo Verde for the IS, LSTM-PC and ConvLSTM-PC models. These sizes were empirically selected. 

In the IS model, the values of correlation, homogeneity, mean and variance from GLCM matrix in four directions (0, 45, 90 and 135 degrees) were used as hand-crafted features. Random Forest with 250 trees and a maximum depth of 25 was used as the classifier. For LSTM, 100 filters were used in the recurrent layer and 100 filters in the FC layer for Campo Verde, while 300 filters were used in the FC layer for Hanover. Network parameters for the ConvLSTM-PC are presented in Table \ref{table:convlstm}. 

For FCN-PL, input patches of size $8 \times 8$ were empirically selected for Hanover as well as of size $32 \times 32$ for Campo Verde. The DenseNet architecture was configured with a growth rate of 16. Average pooling and transpose convolution were used for \textit{TD} and \textit{TU} blocks, and a 20\% dropout was selected. 

Data augmentation was applied to the least represented classes through rotation, and horizontal and vertical flip. Training was made with early stopping regularization. Adam optimizer with learning rate of 0.001 was used for the recurrent networks and Adagrad with 0.01 learning rate for FCN-PL.  
\begin{figure}[t!]
\centering
\includegraphics[scale=0.22]{figs2/ClassOccurrencesUnique.eps}
\caption{Class distribution from Campo Verde dataset. }
\label{fig:class_distr}
\end{figure}

%Two main sequences were identified: First sequence goes from October to Feberuary. Second sequence is between March and July. 