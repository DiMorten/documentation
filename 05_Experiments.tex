\section{Experiments}

%\begin{figure}[t!]
%\label{fig:campoverde}
%\centering
%\includegraphics[scale=0.22]{figs2/CampoVerde.pdf}
%\caption{Campo Verde study area.}
%\label{fig:labels}
%\end{figure}

%\begin{figure}[t!]
%\label{fig:hanover}
%\centering
%\includegraphics[scale=0.18]{figs2/Hanover.pdf}
%\caption{Hanover study area.}
%\label{fig:labels}
%\end{figure}
\subsection{Study Areas}

Agricultural behaviour can greatly vary between different regions. Because of this, two areas with very different weather conditions were studied: First one is located in the surroundings of Hanover city, in Germany. It has an extension of 1728 $km^2$ and consists of a sequence of 24 pre-processed, dual-polarized Sentinel-1 SAR images taken from from October 2014 to October 2016 in a monthly basis. Because this is a temperate region, each parcel belongs to the same class throughout the entire year. Most common crop types are Sugarcane, Soybean and Maize \cite{bargiel2017new}.

Second area is located in Campo Verde municipality from the state of Mato Grosso, Brazil with an extension of 4782 $km^2$. It consists of a sequence of 14 pre-processed, dual-polarized SAR images taken from Sentinel-1 between October 2015 and July 2016. Main classes are \textit{soybean}, \textit{maize} and \textit{cotton}. This area is located in a tropical environment, making its multi-temporal behavior highly dynamic and challenging \cite{sanches2018campo}.

\subsection{Experimental Protocol}

In order to assess the influence of multi-temporal information for crop recognition, the last image from a given sequence was classified using past information from all the images from that sequence.

In each of the studied areas, 50\% of the labeled data was used for training and the remaining 50\% for testing. Standard normalization was applied to the input images. In the case of Campo Verde database, two main sequences were separately studied due to their significant differences in crop distribution (See Figure \ref{fig:class_distr}). Sequence 1 consists of images taken from October 2015 to February 2016, while Sequence 2 happens between March 2016 and July 2016.

The required amount of spatial information differed between studied areas because of their specific crop parcel sizes. For LTM-PC and ConvLSTM-PC, input dimensions and parameter values were set to different values in each case, as presented in Tables \ref{table:lstm} and \ref{table:convlstm}. For FCN-PL, input patches of size 8 were selected for Hanover and size 32 for Campo Verde. The DenseNet architecture was configured with a growth rate of 16 and a dropout of 0.2. In the Image Stacking model, the GLCM features were obtained with windows of 5 for Hanover and 15 for Campo Verde. Random forest was trained with 250 trees and a maximum depth of 25. 

Data augmentation was applied to the least represented classes with rotation; horizontal and vertical flip. Training was made with early stopping regularization. Adam optimizer with learning rate of 0.001 was used for the recurrent networks and Adagrad with 0.01 learning rate for FCN-PL.  
\begin{figure}[t!]
\centering
\includegraphics[scale=0.22]{figs2/ClassOccurrencesUnique.eps}
\caption{Class distribution from Campo Verde dataset. }
\label{fig:class_distr}
\end{figure}

%Two main sequences were identified: First sequence goes from October to Feberuary. Second sequence is between March and July. 